\RequirePackage[l2tabu, orthodox]{nag}
\documentclass[a4paper,draft,12pt,oneside,article,table]{memoir}

%% Geometry
\isopage[10]
\setlength{\topskip}{1.6\topskip} % for \sloppybuttom
\checkandfixthelayout
\sloppybottom

%% Typography
\usepackage{polyglossia,microtype,hyperref,amsmath,unicode-math,xcolor,xspace}
\definecolor{zen-red}{HTML}{B23333}    \definecolor{zen-orange}{HTML}{E57A33}
\definecolor{zen-yellow}{HTML}{F0DFAF} \definecolor{zen-green}{HTML}{5F7F5F}
\definecolor{zen-cyan}{HTML}{93E0E3}   \definecolor{zen-blue}{HTML}{336CB2}
\setdefaultlanguage{english} %polyglossia

\hypersetup{colorlinks=true,linkcolor=zen-red,citecolor=zen-green,urlcolor=zen-orange} % hyperref
\microtypesetup{final,verbose=silent}
\newcommand{\lining}{\setmainfont[Ligatures=TeX,Numbers=Lining]{Arno Pro}} % fontspec
\newcommand{\oldstyle}{\setmainfont[Ligatures=TeX,Numbers=OldStyle]{Arno Pro}} % fontspec
\oldstyle
\setmonofont[Scale=MatchLowercase]{DejaVuSansMono} % fontspec

\unimathsetup{math-style=ISO,bold-style=ISO} % unicode-math
\setmathfont[Scale=MatchLowercase]{Cambria Math} % unicode-math


%% Titlepage
\setlength{\droptitle}{-3em}
\pretitle{\LARGE\par} \posttitle{\vskip 0.5em}
\newcommand{\supertitle}[1]{\gdef\suP{#1}}
\renewcommand{\maketitlehooka}{\ifx\suP\undefined\begin{center}\else\begin{center} {\scshape\suP}\fi}
    \newcommand{\subtitle}[1]{\gdef\suB{#1}}
    \renewcommand{\maketitlehookb}{\ifx\suB\undefined \end{center}\else\par {\large\scshape\suB}\par\end{center}\fi}

%% Header
\newcommand{\stunum}[1]{\gdef\stuN{#1}}
\copypagestyle{articlehead}{plain}
\makeoddhead{articlehead}{\color{gray}\theauthor\ifx\stuN\undefined\else\ifx\stuN\empty\else~(\stuN)\fi\fi}{}{\color{gray}\thedate}
\pagestyle{articlehead}

%% Grapics
\usepackage{tikz,pgfplots,tikz-timing,rotating}
\usetikzlibrary{mindmap,calc,arrows,positioning,shapes,matrix}

%% TÅGEKAMMERET
% \usepackage{tket,tkvc}
% \newfontface\bbface[Scale=0.87]{TeX Gyre Termes Math} \TKsetup{C = {\bbface\kern-0.1exℂ}} % fontspec,tket

%% resten
\usepackage{threeparttable,siunitx,pdfpages,algpseudocode,algorithm}
\sisetup{per-mode=symbol}
\makeatletter \renewcommand{\ALG@name}{Algoritme}\makeatother
\usepackage{minted} % requires minted > 2.0-alpha2
\usemintedstyle{tango}

\newcommand{\srcpath}{../ex09/src/main/java/ddist}
\newcommand{\inmnt}[3]{\vspace{1em}\noindent\texttt{\color{gray}File: #3}\vspace{-1em}\inputminted[tabsize=4,firstline=#1,firstnumber=#1,lastline=#2,linenos]{java}{\srcpath/#3}}
\newcommand{\mil}[1]{\mintinline{java}{#1}}

%% Help
\usepackage{lipsum}
\usepackage[margin,draft]{fixme} \fxusetheme{color}

\usepackage{ragged2e}

\usepackage{csquotes}
\usepackage[
  language=english,
  backend=biber,
  sortcites=true,
  hyperref=true,
]{biblatex}
\addbibresource{literatur.bib}


\newcommand*{\prtA}{\textsc{a}\xspace}
\newcommand*{\prtB}{\textsc{b}\xspace}
\newcommand*{\textex}[1]{\ensuremath{\text{\texttt{#1}}}\xspace}
\newcommand*{\opname}[1]{\ensuremath{\text{\textit{#1}}}\xspace}
\newcommand*{\state}[1]{{\lining#1}\xspace}
\DeclareMathOperator{\transform}{transform}

\begin{document}
\supertitle{Distributed Systems}
\title{Exercise 11}
% \subtitle{}
\author{Richard~Möhn~\small{(201311231)} \and Mathias~Dannesbo~\small{(201206106)}}
% \stunum{201311231, 201206106}
\date{\today}
\maketitle

\chapter{Introduction}
We used pairprogramming for all the code and ``pairreporting'' for the
report, so we share the workload at 50\% each.

In this report we present an editor, the instances of which are able to
connect to each other and to capture editing actions in one editor and
replay them in another \textit{in the same text}.  We built upon the
editor from excercise~09, which was able to connect and send edit event
to other instances of itself, but not in the same textarea. Our
contribution was to make two editors merge their edits in the same
text.

The report describes how we changed the architecture of the event flow
based on operational transformation. It discusses some of the
decisions we made in developing the editor. The conclusion contains a
list of issues with the editor that still need to be addressed.

\chapter{Operational transformation}

There exist quite some applications for collaborative text editing;
examples are Saros \cite{Saros}, Google Wave \cite{GoogleWave} or Gobby
\cite{Gobby}. They all use a client-server approach for synchronising
optimistically. This means, changes to the text are made immediately
(when the user presses a key, for example) and then sent to the server,
which in turn sends them to the other clients. If changes are made
concurrently by multiple users, an algorithm makes sure that everyone
ends up with the same text. This approach to synchronisation, or to
preserving consistency, increases distribution transparency since there
is no delay between user input and its effects on the screen, just as in
a non-distributed editor. 

One method for this kind of synchronisation is operational
transformation: Operations represent changes to the text and they are
first applied locally, then sent to the other parties editing the same
text. If the other parties edit their versions of the text in the
meantime, they have to transform the received operation so that it does
what was intended.

Example: \prtA and \prtB start with text \textex{abcdefg}. \prtA inserts
\textex{123} at offset 2 and gets \textex{ab123cdefg}, then sends the
operation to \prtB. Before \prtB receives it, she removes two characters
at offset 4 and gets \textex{abcdg}, then she sends this operation to
\prtA.  Now, \prtA and \prtB receive the operations from each other, but
if they applied them directly, they would end up with different texts:
\prtA with \textex{ab12defg} and \prtB with \textex{ab123cdg}. In this
case, the result \prtB gets already appears like a sensible combination
of the two operations. Then only the remove operation \prtA received has
to be shifted by three characters to the right, so that it ends up with
that text, too.  Mathematically:

\begin{align*}
    t               &:= \textex{abcdefg} \\
    o_A             &:= (\opname{insert}, 2, \textex{123}) \\
    o_B             &:= (\opname{remove}, 4, 2) \\
    o_A(t)          &= \textex{ab123cdefg} \\
    o_B(t)          &= \textex{abcdg} \\
    (o_B ∘ o_A)(t)  &= \textex{ab12defg} \\
    (o_A ∘ o_B)(t)  &= \textex{ab123cdg} \\
    (o_A', o_B')    &:= \transform(o_A, o_B) \\
    o_A'            &= (\opname{insert}, 2, \textex{123}) \\
    o_B'            &= (\opname{remove}, 4, 5) \\
    (o_B' ∘ o_A)(t) &= (o_A' ∘ o_B)(t) = \textex{ab123cdg} \\
\end{align*}

The $\transform$ function is the first component of operational
transformation. Its defining property is: 

\begin{equation*}
    ∀ o_a, o_b: (o_a', o_b') := \transform(o_a, o_b) ⇒ (o_b' ∘ o_a) = (o_a' ∘ o_b)
\end{equation*}

That means, if two copies of the same text have diverged by applying
$o_a$ to one and $o_b$ to the other, then applying $o_b'$ after $o_a$
and $o_a'$ after $o_b$ brings them in synch again. Any function that has
the above property can be used as $\transform$, but most of them will
only give consistent, but not sensible, results.

The second component of operational transformation is an algorithm that
wraps the $\transform$ function, sends operations to other parties,
receives operations from other parties, transforms them and applies them
locally.  All the applications mentioned above use a version of the
so-called Jupiter algorithm, which was originally published in
\cite{Jupiter}. By using a client-server approach it reduces the problem
of synchronising $n$ parties to the problem of synchronising 2 parties.

\tikzset{ global/.style={draw=black,line width=1pt, inner sep=.5em,
    minimum height=2em, minimum width=11em, text centered},
  c/.style={global, cloud, aspect=2, cloud puffs=16,},
  n/.style={global, rectangle},
  r/.style={n, fill=red!90!yellow!20},
  g/.style={n, fill=green!20},
  b/.style={n, fill=blue!20},
  y/.style={n, fill=yellow!20},
  l/.style={>=latex',line width=1.5pt},
  sa/.style={l,->,shorten >=1pt},
  st/.style={sa, red!90!black, text=black},
  sj/.style={sa, blue!70!black, text=black,},
  sb/.style={sa, shorten <=1pt,black!30!white!90!red},
  jm/.style={sa, line width=1pt,shorten >=.35cm,shorten <=.35cm},
  jr/.style={jm, red!90!black, text=black},
  jb/.style={jm, blue!70!black, text=black, },
  jne/.style={shift={(2pt,2pt)}},
  jnw/.style={shift={(-2pt,2pt)}},
  jsw/.style={shift={(-2pt,-2pt)}},
  jse/.style={shift={(2pt,-2pt)}},
}
\begin{figure}[htb]
  \centering
  \caption{The state space in the Jupiter algorithm.}
  \lining
  \begin{tikzpicture}
    \clip (.5,-3) rectangle (8.25,3.75); % this
    \begin{turn}{-45}                    % and this is black-magic. 
      \begin{scope}[rotate=45,row sep=0.5cm, column sep=0.5cm,nodes={rotate=45},]

      \matrix (jup) [matrix of nodes] {
        0,0 & 0,1 & 0,2 & 0,3 \\
        1,0 & 1,1 & 1,2 & 1,3 \\
        2,0 & 2,1 & 2,2 & 2,3 \\
        3,0 & 3,1 & 3,2 & 3,3 \\
      };

      \draw[jr] ([jse]jup-1-1.center) -- ([jse]jup-2-1.center);
      \draw[jb] ([jnw]jup-1-1.center) -- ([jnw]jup-2-1.center);

      \draw[jr] ([jne]jup-2-1.center) -- ([jne]jup-2-2.center);
      \draw[jb] ([jsw]jup-2-1.center) -- ([jsw]jup-2-2.center);

      \draw[jb] (jup-2-2.center) -- (jup-3-2.center);

      \draw[jr] (jup-2-2.center) -- (jup-2-3.center);

      \draw[jb] (jup-3-2.center) -- (jup-3-3.center);

      \draw[jr] (jup-2-3.center) -- (jup-3-3.center);

      \draw[jr] ([jne]jup-3-3.center) -- ([jne]jup-3-4.center);
      \draw[jb] ([jsw]jup-3-3.center) -- ([jsw]jup-3-4.center);

      \draw[jb, line width=1.5pt] (-1,3.5) --node[above left]{client} (-3,1.5);
      \draw[jr, line width=1.5pt] (1,3.5) --node[above right]{server} (3,1.5);
      \end{scope}
    \end{turn}
  \end{tikzpicture}
  \oldstyle
  \label{fig:state}
\end{figure}

\cite{Jupiter} and \cite{UnderstandOT} contain extensive descriptions of
how it works, so we won't go into details here. However, figure
\ref{fig:state} illustrates its workings. The numbers count the
operations applied to the text on the client and server and, paired up,
they are logical timestamps identifying the states the edited text goes
through. The arrows correspond to operations applied to the text, the
state of which changes after every application of an operation.
Operations include the timestamp of the state to which they were
applied. Both the client and the server start in state \state{0,0}.
First, the client generates and applies an operation and the text goes
into state \state{1,0}. The client sends the operation to the server,
which applies it, so that its copy of the text goes into state
\state{1,0}, too. The same happens the other way around when the server
generates the second operation. Then the client applies an operation and
the also server applies an operation, before receiving that of the
client. The result are two different states of the text after the same
total number of operations. Client and server now receive each other's
operations, but cannot apply them because their texts are not in the
state to which the operations where originally applied. In order to get
from their diverged states into a common state again, client and server
have to transform the received operations and apply the results. Then
they proceed without conflicts.

This is most of what the Jupiter algorithm does. It gets only slightly
more difficult when client and server diverge by more than one
operation.


\chapter{Code Overview}
\begin{figure}[htb]
  \centering
  \caption{\mil{Event}'s path through the system in the editor as
    provided and in our version. \protect\tikz[baseline=-0.5ex]\protect\draw[st] (0,0) -- (0.5,0); denote \mil{TextChangeEvents} and \protect\tikz[baseline=-0.5ex]\protect\draw[sj] (0,0) -- (0.5,0); denote \mil{JupiterEvents}.}
  \begin{tikzpicture}[node distance=1cm, auto]
    \footnotesize
    \node[c] (net) {Network};
    \node[b, above right=2.5em of net.north] (esen) {\texttt{EventSender}};
    \node[g, above=2.5em of esen] (oqueue) {Outbound queue};
    \node[r, above=12em of net] (jc) {\texttt{JupiterClient}};
    \node[b, above left=2.5em of net.north] (erec) {\texttt{EventReceiver}};
    \node[g, above=2.5em of erec] (iqueue) {Inbound queue};
    \node[g, above=2.5em of jc.north east] (dqueue) {Display queue};
    \node[r, above=2.5em of dqueue] (edis) {\texttt{EventDisplayer}};
    \node[r, left=2.5em of jc] (dec) {};

    \node[r, above=2.5em of dec] (ta) {Text area};

    \draw[sj] (net.north west) --  (erec.south);
    \draw[sj] (erec.north) --  (iqueue.south);
    \draw[st] (iqueue.20) --  (jc.200);
    \draw[sj,dashed] (iqueue.20) --  (jc.200);
    \draw[sj] (jc.340) --  (oqueue.160);
    \draw[sj] (oqueue.south) --  (esen.north);
    \draw[sj] (esen.south) --  (net.north east);
    \draw[st] (dec.320) --  (iqueue.160);

    \draw[st] (jc.20) --  (dqueue.230);
    \draw[st] (dqueue.north) --  (edis.south);

    \draw[sa] (edis.south west) --  (ta.north east);

    \draw[sa] (ta.220) --  (dec.140);

    \draw[sa] (ta.347) --  (dec.13);


    \draw[sa] (dec.35) --  (ta.325);

    \draw[sb,out=290,in=130,] (dec.140) to  (dec.320);
    \path[sb] (dec.13) edge[bend left=45]  (dec.35);
    \path[sb] (ta.north east) edge[bend right=25]  (ta.347);
    \node[n, left=2.5em of jc] (dec2) {\texttt{DocumentEventCapturer}};
  \end{tikzpicture}
  \label{fig:event}
\end{figure}

Thanks to the layered architecture (not very stringent) and message
queueing, we didn't have to change much of the existing code in order to
allow for distributed editing of the same text. -- Of course, one text
area had to go and \mil{DocumentEventCapturer} and \mil{EventDisplayer}
(formerly \mil{EventReplayer}) now operate on the remaining one.
But otherwise only the \mil{JupiterClient} got plugged in between
receiving and sending, and capturing and displaying events,
respectively. Figure \ref{fig:event} shows the new configuration.


\section{\mil{JupiterClient}}
%\inmnt{35}{80}{JupiterClient.java} -- I think I won't need this.

The \mil{JupiterClient} is called like that because it will be used as
the client Jupiter class for the next handin with $n$ parties. Right
now, there is one \mil{JupiterClient}s on each of the two editors
running.

All events go through the \mil{JupiterClient}: Both the user inputs in the
form of \mil{TextChangeEvent}s and events from the other
\mil{JupiterClient} in the form of \mil{JupiterEvent}s come in through
the inbound queue.  If an incoming event came from the user's actions on
the text area, the \mil{JupiterClient} wraps it in a \mil{JupiterEvent},
including a timestamp and sends it to the other editor. It gives it
to the \mil{EventReplayer}, so that the user actually sees the result of
her input. If, on the other hand, an incoming event came from the other
editor, the \mil{JupiterClient} transforms it, if necessary, before
giving it to the \mil{EventReplayer}.


\section{\mil{DocumentEventCapturer}}
It is not possible to send the text directly to the text area from the
\mil{EventDisplayer} without it going through
\mil{DocumentEventCapturer}, witch extends \mil{DocumentFilter}. To
circumvent this we put a switch in \mil{DocumentEventCapturer}.

Before \mil{EventDisplayer} send the text to the text area, it set the
\mil{isGenerateEvents} to \mil{false} to change the function of
\mil{DocumentEventCapturer}. When it is set it does not generate
event, but instead add text to the area. Afterwards it it set to
\mil{true} again.

The gray arrows on the \mil{DocumentEventCapturer} on figure
\ref{fig:event} represent this switch.  The listing below shows how
the switch is implementet in the \mil{insertString}-method in
\mil{DocumentEventCapturer}.

\inmnt{41}{51}{DocumentEventCapturer.java}

\section{\mil{Transformer}}

As mentioned earlier, the transformation function can be implemented in
different ways. The Saros project, for example, uses what is described
in \cite{Transform}. We decided for a more primitive approach, which
covers all cases, but does not always yield satisfactory results.
However, we wrapped the function \mil{transform} in a class
\mil{Transform}, so that the structure can easily be changed to match
the strategy pattern. Thereby, different solutions to the same problem
could be plugged in without effort.

\fxwarning{Possibly needs to be adapted to the text. Or just include an
ASCII art thing like I drew in the Transformer.}
\begin{figure}[htb]
\centering
\caption{Transformation of two overlapping remove operations}
\label{fig:transform}
\newcommand\ncoord[2][0,0]{\tikz[remember picture,overlay]{\path (#1) coordinate (#2);}}
\newcommand\acoord[2][.3em]{\ncoord[#1,.8em]{#2}}
\newcommand\remove[4]{\filldraw[#1] ($(#2) +(.15em,0em)$) -- ($(#3) +(-.15em,0em)$) -- ++(0em,-.6em) to [out=-90,in=90] ($(#4) + (.1em,0em)$) |- ++(-.2em,-.8em) --++(0em,.8em) to [out=90,in=-90] ($(#2) + (.15em,-.8em)$) --cycle;}
%\texttt{… a b c d e f g h i j k …}\\
\ttfamily
… a b c\acoord{1cd} d e f\acoord{1fg} g\acoord{1gh} h i\acoord{1ij} j k …\\
\vspace{4em}
… a b c\acoord{2ch} h i\acoord{2ij} j k … \quad
… a b c\acoord{3cd} d e f\acoord{3fj} j k …\\
\vspace{4em}
… a b c\acoord[.2em]{4cjl}\acoord[.4em]{4cjr} j k …

\tikzset{
  li/.style={opacity=.15,draw opacity=.5,},
  dk/.style={opacity=.3,draw opacity=.7,},
}
\tikz[overlay,remember picture] {
    %\draw[thick,blue] (1g) -- ++(0,-0.1em) -|  (1I);
  \remove{blue,dk}{1fg}{1ij}{3fj}
  \remove{blue,li}{2ch}{2ij}{4cjl}
  \remove{red,dk}{1cd}{1gh}{2ch}
  \remove{red,li}{3cd}{3fj}{4cjr}
}
\end{figure}

As an example, figure \ref{fig:transform} shows what \mil{transform}
does for two overlapping remove events: On the local machine the smaller
removal to the right and on the remote machine the larger removal to the
left have happened. In this case, transforming the operations amounts to
make them delete less and possibly in a different place. The following
listing shows the piece of source code that covers this and some other
cases. The last two assignments are the transformed operations in this
case.

\inmnt{149}{184}{Transformer.java}

\chapter{Conclusion}

We had an editor whose instances could connect to each other and where
the user could see what the other was doing to her text. Now we extended
this, enabling both users to edit the same text at the same time. For
synchronisation we use the well-known Jupiter algorithm, one advantage
of which is that users see their changes without delay.

As usual, some issues, mainly in regard of user friendliness remain. For
instance, there is no indication whether editors are connected. However,
those things are in no way related to distributed systems and therefore
have to stand back behind things that are. Another example is the
relative primitivity of the \mil{transform} function. This is also not
severe, since it only very rarely, if at all, affects sensible text
editing actions.

All in all, the architecture created for the previous version of the
editor proved quite successful in letting us concentrate on the
synchronisation issues central to this exercise. And using the exising
Jupiter algorithm is, in our opinion, a much better approach than
inventing some half-baked own synchronisation method.

\appendix

\chapter{Finding the Code and Running the Editor}

The file \texttt{Code1864-ex11.zip} contains a Maven repository with the source
code and a \textsc{jar} file being the executable editor. From the
root directory it can be run with \texttt{./run.sh}.

% \clearpage
% \listoftables
% \listoffigures
% \listoflistings
% \nocite{*}
% \bibliographystyle{dlfltxbbibtex} \Bibliography{bib}
% \clearpage \appendix

\setlength{\RaggedRightRightskip}{0pt plus 4em} % maybe
\RaggedRight
\fxwarning{BibLaTeX might be displaying too much information. Just throw
them out of literatur.bib.}
\printbibliography

\end{document}

%%% Local Variables:
%%% coding: utf-8
%%% mode: latex
%%% TeX-engine: xetex
%%% End:
