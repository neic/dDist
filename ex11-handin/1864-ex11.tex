\RequirePackage[l2tabu, orthodox]{nag}
\documentclass[a4paper,draft,12pt,oneside,article,table]{memoir}

%% Geometry
\isopage[10]
\setlength{\topskip}{1.6\topskip} % for \sloppybuttom
\checkandfixthelayout
\sloppybottom

%% Typography
\usepackage{polyglossia,microtype,hyperref,amsmath,unicode-math,xcolor,natbib}
\definecolor{zen-red}{HTML}{B23333}    \definecolor{zen-orange}{HTML}{E57A33}
\definecolor{zen-yellow}{HTML}{F0DFAF} \definecolor{zen-green}{HTML}{5F7F5F}
\definecolor{zen-cyan}{HTML}{93E0E3}   \definecolor{zen-blue}{HTML}{336CB2}
\setdefaultlanguage{english} %polyglossia

\hypersetup{colorlinks=true,linkcolor=zen-red,citecolor=zen-green,urlcolor=zen-orange} % hyperref
\microtypesetup{final,verbose=silent} 
\setmainfont[Ligatures=TeX,Numbers=OldStyle]{Arno Pro} % fontspec
\setmonofont[Scale=MatchLowercase]{DejaVuSansMono} % fontspec

\unimathsetup{math-style=ISO,bold-style=ISO} % unicode-math
\setmathfont[Scale=MatchLowercase]{Cambria Math} % unicode-math


%% Titlepage
\setlength{\droptitle}{-3em}
\pretitle{\LARGE\par} \posttitle{\vskip 0.5em}
\newcommand{\supertitle}[1]{\gdef\suP{#1}}
\renewcommand{\maketitlehooka}{\ifx\suP\undefined\begin{center}\else\begin{center} {\scshape\suP}\fi}
\newcommand{\subtitle}[1]{\gdef\suB{#1}}
\renewcommand{\maketitlehookb}{\ifx\suB\undefined \end{center}\else\par {\large\scshape\suB}\par\end{center}\fi}

%% Header
\newcommand{\stunum}[1]{\gdef\stuN{#1}}
\copypagestyle{articlehead}{plain}
\makeoddhead{articlehead}{\color{gray}\theauthor\ifx\stuN\undefined\else\ifx\stuN\empty\else~(\stuN)\fi\fi}{}{\color{gray}\thedate}
\pagestyle{articlehead}

%% Grapics
\usepackage{tikz,pgfplots,tikz-timing}
\usetikzlibrary{mindmap,arrows,positioning,shapes}

%% TÅGEKAMMERET
%\usepackage{tket,tkvc}
%\newfontface\bbface[Scale=0.87]{TeX Gyre Termes Math} \TKsetup{C = {\bbface\kern-0.1exℂ}} % fontspec,tket

%% resten
\usepackage{threeparttable,siunitx,pdfpages,algpseudocode,algorithm}
\sisetup{per-mode=symbol}
\makeatletter \renewcommand{\ALG@name}{Algoritme}\makeatother
\usepackage{minted} % requires minted > 2.0-alpha2
\usemintedstyle{tango}

\newcommand{\srcpath}{../ex09/src/main/java/ddist}
\newcommand{\inmnt}[3]{\vspace{1em}\noindent\texttt{\color{gray}File: #3}\vspace{-1em}\inputminted[tabsize=4,firstline=#1,firstnumber=#1,lastline=#2,linenos]{java}{\srcpath/#3}}
\newcommand{\mil}[1]{\mintinline{java}{#1}}

%% Help
\usepackage{lipsum}
\usepackage[margin,draft]{fixme} \fxusetheme{color}

\begin{document}
\supertitle{Distributed Systems}
\title{Exercise 11}
%\subtitle{}
\author{Richard~Möhn~\small{(201311231)} \and Mathias~Dannesbo~\small{(201206106)}}
%\stunum{201311231, 201206106}
\date{\today}
\maketitle


\tikzset{ global/.style={draw=black,line width=1pt, inner sep=.5em,
    minimum height=2em, minimum width=11em, text centered},
  c/.style={global, cloud, aspect=2, cloud puffs=16,},
  n/.style={global, rectangle},
  r/.style={n, fill=red!20},
  g/.style={n, fill=green!20},
  b/.style={n, fill=blue!20},
  y/.style={n, fill=yellow!20},
  l/.style={>=latex',line width=1.5pt},
  sa/.style={l,->,shorten >=1pt},
}
\begin{figure}[hbp]
  \centering
  \caption{\mil{MyTextEvent}'s path through the system in the editor as
  provided and in our version.}
  \begin{tikzpicture}[node distance=1cm, auto]
    \footnotesize
    \node[c] (net) {Network};
    \node[b, above right=2.5em of net.north] (esen) {\texttt{EventSender}};
    \node[g, above=2.5em of esen] (oqueue) {Outbound queue};
    \node[r, above=12em of net] (jc) {\texttt{JupterClient}};
    \node[b, above left=2.5em of net.north] (erec) {\texttt{EventReceiver}};
    \node[g, above=2.5em of erec] (iqueue) {Inbound queue};
    \node[g, above right=2.5em of jc.north] (dqueue) {Display queue};
    \node[r, above=2.5em of dqueue] (edis) {\texttt{EventDisplayer}};
    \node[r, left=2.5em of jc] (dec) {\texttt{DocumentEventCapturer}};

    \node[r, above=2.5em of dec] (ta) {Text area};

%    \node[above=1em of dec] (aft) {Network enabled};

    \draw[sa] (net.north west) --  (erec.south);
    \draw[sa] (erec.north) --  (iqueue.south);
    \draw[sa] (iqueue.20) --  (jc.200);
    \draw[sa] (jc.340) --  (oqueue.160);
    \draw[sa] (oqueue.south) --  (esen.north);
    \draw[sa] (esen.south) --  (net.north east);
    \draw[sa] (dec.340) --  (iqueue.160);

    \draw[sa] (jc.20) --  (dqueue.200);
    \draw[sa] (dqueue.north) --  (edis.south);

    \draw[sa] (edis.west) --  (dec.north east);

    \draw[sa] (ta.200) --  (dec.160);

    \draw[sa] (dec.20) --  (ta.340);

    % \node[right=0.3 of dec.north east] (a1) {};
    % \node[right=0.3 of esen.south east] (a2) {};
    % \draw[line width=1pt] (a1) -- (a2);
    % \node[right=0.5 of oqueue] (a3) {Sending editor};

    % \node[right=0.3 of erec.north east] (b1) {};
    % \node[right=0.3 of erep2.south east] (b2) {};
    % \draw[line width=1pt] (b1) -- (b2);
    % \node[right=0.5 of iqueue] (b3) {Receiving editor};
  \end{tikzpicture}
  \label{fig:event}
\end{figure}


\chapter{Conclusion}


\appendix

\chapter{Finding the Code and Running the Editor}

The file \texttt{Code1864-ex09.zip} contains a Maven repository with the source
code and a \textsc{jar} file being the executable editor. From the
root directory it can be run with \texttt{./run.sh}. The code can also
be found online at \url{wiply.neic.dk/au/ddist/Code1864-ex09.zip}.

%\clearpage
% \listoftables
% \listoffigures
% \listoflistings
%\nocite{*}
%\bibliographystyle{dlfltxbbibtex} \Bibliography{bib}
%\clearpage \appendix

\end{document}

%%% Local Variables:
%%% coding: utf-8
%%% mode: latex
%%% TeX-engine: xetex
%%% End:
