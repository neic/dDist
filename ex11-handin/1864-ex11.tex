\RequirePackage[l2tabu, orthodox]{nag}
\documentclass[a4paper,draft,12pt,oneside,article,table]{memoir}

%% Geometry
\isopage[10]
\setlength{\topskip}{1.6\topskip} % for \sloppybuttom
\checkandfixthelayout
\sloppybottom

%% Typography
\usepackage{polyglossia,microtype,hyperref,amsmath,unicode-math,xcolor,natbib}
\definecolor{zen-red}{HTML}{B23333}    \definecolor{zen-orange}{HTML}{E57A33}
\definecolor{zen-yellow}{HTML}{F0DFAF} \definecolor{zen-green}{HTML}{5F7F5F}
\definecolor{zen-cyan}{HTML}{93E0E3}   \definecolor{zen-blue}{HTML}{336CB2}
\setdefaultlanguage{english} %polyglossia

\hypersetup{colorlinks=true,linkcolor=zen-red,citecolor=zen-green,urlcolor=zen-orange} % hyperref
\microtypesetup{final,verbose=silent}
\newcommand{\lining}{\setmainfont[Ligatures=TeX,Numbers=Lining]{Arno Pro}} % fontspec
\newcommand{\oldstyle}{\setmainfont[Ligatures=TeX,Numbers=OldStyle]{Arno Pro}} % fontspec
\oldstyle
\setmonofont[Scale=MatchLowercase]{DejaVuSansMono} % fontspec

\unimathsetup{math-style=ISO,bold-style=ISO} % unicode-math
\setmathfont[Scale=MatchLowercase]{Cambria Math} % unicode-math


%% Titlepage
\setlength{\droptitle}{-3em}
\pretitle{\LARGE\par} \posttitle{\vskip 0.5em}
\newcommand{\supertitle}[1]{\gdef\suP{#1}}
\renewcommand{\maketitlehooka}{\ifx\suP\undefined\begin{center}\else\begin{center} {\scshape\suP}\fi}
    \newcommand{\subtitle}[1]{\gdef\suB{#1}}
    \renewcommand{\maketitlehookb}{\ifx\suB\undefined \end{center}\else\par {\large\scshape\suB}\par\end{center}\fi}

%% Header
\newcommand{\stunum}[1]{\gdef\stuN{#1}}
\copypagestyle{articlehead}{plain}
\makeoddhead{articlehead}{\color{gray}\theauthor\ifx\stuN\undefined\else\ifx\stuN\empty\else~(\stuN)\fi\fi}{}{\color{gray}\thedate}
\pagestyle{articlehead}

%% Grapics
\usepackage{tikz,pgfplots,tikz-timing,rotating}
\usetikzlibrary{mindmap,calc,arrows,positioning,shapes,matrix}

%% TÅGEKAMMERET
% \usepackage{tket,tkvc}
% \newfontface\bbface[Scale=0.87]{TeX Gyre Termes Math} \TKsetup{C = {\bbface\kern-0.1exℂ}} % fontspec,tket

%% resten
\usepackage{threeparttable,siunitx,pdfpages,algpseudocode,algorithm}
\sisetup{per-mode=symbol}
\makeatletter \renewcommand{\ALG@name}{Algoritme}\makeatother
\usepackage{minted} % requires minted > 2.0-alpha2
\usemintedstyle{tango}

\newcommand{\srcpath}{../ex09/src/main/java/ddist}
\newcommand{\inmnt}[3]{\vspace{1em}\noindent\texttt{\color{gray}File: #3}\vspace{-1em}\inputminted[tabsize=4,firstline=#1,firstnumber=#1,lastline=#2,linenos]{java}{\srcpath/#3}}
\newcommand{\mil}[1]{\mintinline{java}{#1}}

%% Help
\usepackage{lipsum}
\usepackage[margin,draft]{fixme} \fxusetheme{color}

\begin{document}
\supertitle{Distributed Systems}
\title{Exercise 11}
% \subtitle{}
\author{Richard~Möhn~\small{(201311231)} \and Mathias~Dannesbo~\small{(201206106)}}
% \stunum{201311231, 201206106}
\date{\today}
\maketitle

\chapter{Introduction}
We used pairprogramming for all the code and ``pairreporting'' for the
report, so we share the workload at 50\% each.

In this report we present an editor, the instances of which are able
to connect to each other and to capture editing actions in one editor
and replay them in another \textit{in the same textarea}.  We build
upon the editor from excercise~09. Whitch where able to connect and
send edit event to each other, but not in the same textarea. Our contribution was to
make two editors merge their edit in the same window.

The report describes how we change the architechture of the event flow
based on operational transformation. It discusses some of the
decisions we made in developing the editor. The conclusion contains a
list of issues with the editor that still need to be addressed.

\chapter{Operational transformation}
\fxwarning{Richard writes something about the techniques.}

\tikzset{ global/.style={draw=black,line width=1pt, inner sep=.5em,
    minimum height=2em, minimum width=11em, text centered},
  c/.style={global, cloud, aspect=2, cloud puffs=16,},
  n/.style={global, rectangle},
  r/.style={n, fill=red!90!yellow!20},
  g/.style={n, fill=green!20},
  b/.style={n, fill=blue!20},
  y/.style={n, fill=yellow!20},
  l/.style={>=latex',line width=1.5pt},
  sa/.style={l,->,shorten >=1pt},
  st/.style={sa,orange},
  sj/.style={sa,green!70!black},
  sb/.style={sa, shorten <=1pt,black!30!white!90!red},
  jm/.style={sa, line width=1pt,shorten >=.35cm,shorten <=.35cm},
  jr/.style={jm, red!90!black, text=black},
  jb/.style={jm, blue!70!black, text=black, },
  jne/.style={shift={(2pt,2pt)}},
  jnw/.style={shift={(-2pt,2pt)}},
  jsw/.style={shift={(-2pt,-2pt)}},
  jse/.style={shift={(2pt,-2pt)}},
}

\begin{figure}[hbp]
  \centering
  \caption{The state space in the Jupiter algorithm.}
  \lining
  \begin{turn}{-45}
    \begin{tikzpicture}[rotate=45,row sep=0.5cm, column sep=0.5cm,nodes={rotate=45},]

      \matrix (jup) [matrix of nodes]
      {
        0,0 & 0,1 & 0,2 & 0,3 \\
        1,0 & 1,1 & 1,2 & 1,3 \\
        2,0 & 2,1 & 2,2 & 2,3 \\
        3,0 & 3,1 & 3,2 & 3,3 \\
      };

      \draw[jr] ([jse]jup-1-1.center) -- ([jse]jup-2-1.center);
      \draw[jb] ([jnw]jup-1-1.center) -- ([jnw]jup-2-1.center);

      \draw[jr] ([jne]jup-2-1.center) -- ([jne]jup-2-2.center);
      \draw[jb] ([jsw]jup-2-1.center) -- ([jsw]jup-2-2.center);

      \draw[jb] (jup-2-2.center) -- (jup-3-2.center);

      \draw[jr] (jup-2-2.center) -- (jup-2-3.center);

      \draw[jb] (jup-3-2.center) -- (jup-3-3.center);

      \draw[jr] (jup-2-3.center) -- (jup-3-3.center);

      \draw[jr] ([jne]jup-3-3.center) -- ([jne]jup-3-4.center);
      \draw[jb] ([jsw]jup-3-3.center) -- ([jsw]jup-3-4.center);

      \draw[jb, line width=1.5pt] (-1,3.5) --node[above left]{client} (-3,1.5);
      \draw[jr, line width=1.5pt] (1,3.5) --node[above right]{server} (3,1.5);

    \end{tikzpicture}
  \end{turn}
  \oldstyle
  \label{fig:state}
\end{figure}



\chapter{Code Overview}
Our main addition of code was in and around the \mil{JupiterClient}-class. We changed the flow of the event to go trough it. It is most easily show on figure \ref{fig:event}.
\fxwarning{Richard writes something.}

\begin{figure}[hbp]
  \centering
  \caption{\mil{Event}'s path through the system in the editor as
    provided and in our version. \protect\tikz[baseline=-0.5ex]\protect\draw[st] (0,0) -- (0.5,0); denote \mil{TextEditEvents} and \protect\tikz[baseline=-0.5ex]\protect\draw[sj] (0,0) -- (0.5,0); denote \mil{JupiterEvents}.}
  \begin{tikzpicture}[node distance=1cm, auto]
    \footnotesize
    \node[c] (net) {Network};
    \node[b, above right=2.5em of net.north] (esen) {\texttt{EventSender}};
    \node[g, above=2.5em of esen] (oqueue) {Outbound queue};
    \node[r, above=12em of net] (jc) {\texttt{JupiterClient}};
    \node[b, above left=2.5em of net.north] (erec) {\texttt{EventReceiver}};
    \node[g, above=2.5em of erec] (iqueue) {Inbound queue};
    \node[g, above=2.5em of jc.north east] (dqueue) {Display queue};
    \node[r, above=2.5em of dqueue] (edis) {\texttt{EventDisplayer}};
    \node[r, left=2.5em of jc] (dec) {};


    \node[r, above=2.5em of dec] (ta) {Text area};

    \draw[sj] (net.north west) --  (erec.south);
    \draw[sj] (erec.north) --  (iqueue.south);
    \draw[st] (iqueue.20) --  (jc.200);
    \draw[sj,dashed] (iqueue.20) --  (jc.200);
    \draw[sj] (jc.340) --  (oqueue.160);
    \draw[sj] (oqueue.south) --  (esen.north);
    \draw[sj] (esen.south) --  (net.north east);
    \draw[st] (dec.320) --  (iqueue.160);

    \draw[st] (jc.20) --  (dqueue.230);
    \draw[st] (dqueue.north) --  (edis.south);

    \draw[sa] (edis.south west) --  (dec.north east);

    \draw[sa] (ta.220) --  (dec.140);

    \draw[sa] (dec.20) --  (ta.340);

    \draw[sb,out=290,in=130,] (dec.140) to  (dec.320);
    \path[sb] (dec.north east) edge[bend left=45]  (dec.20);
    \node[n, left=2.5em of jc] (dec2) {\texttt{DocumentEventCapturer}};
  \end{tikzpicture}
  \label{fig:event}
\end{figure}

\section{\mil{JupiterClient}}
%\inmnt{35}{80}{JupiterClient.java}
\fxwarning{Richard writes something.}

\section{\mil{DocumentEventCapturer}}
%\inmnt{195}{228}{DocumentEventCapturer.java}
\fxwarning{Mathias writes something.}


\section{\mil{Transformer}}
\fxwarning{Richard writes something.}
%\inmnt{195}{228}{Transformer.java}
\newcommand\ncoord[2][0,0]{\tikz[remember picture,overlay]{\path (#1) coordinate (#2);}}
\newcommand\acoord[2][.3em]{\ncoord[#1,.8em]{#2}}
\newcommand\remove[4]{\filldraw[#1] ($(#2) +(.15em,0em)$) -- ($(#3) +(-.15em,0em)$) -- ++(0em,-.6em) to [out=-90,in=90] ($(#4) + (.1em,0em)$) |- ++(-.2em,-.8em) --++(0em,.8em) to [out=90,in=-90] ($(#2) + (.15em,-.8em)$) --cycle;}
\newpage
%\texttt{… a b c d e f g h i j k …}\\
\begin{center}
\ttfamily
… a b c\acoord{1cd} d e f\acoord{1fg} g\acoord{1gh} h i\acoord{1ij} j k …\\
\vspace{4em}
… a b c\acoord{2ch} h i\acoord{2ij} j k … \quad
… a b c\acoord{3cd} d e f\acoord{3fj} j k …\\
\vspace{4em}
… a b c\acoord[.2em]{4cjl}\acoord[.4em]{4cjr} j k …

\tikzset{
  li/.style={opacity=.15,draw opacity=.5,},
  dk/.style={opacity=.3,draw opacity=.7,},
}
\tikz[overlay,remember picture] {
    %\draw[thick,blue] (1g) -- ++(0,-0.1em) -|  (1I);
  \remove{blue,dk}{1fg}{1ij}{3fj}
  \remove{blue,li}{2ch}{2ij}{4cjl}
  \remove{red,dk}{1cd}{1gh}{2ch}
  \remove{red,li}{3cd}{3fj}{4cjr}
}
\end{center}

\chapter{Conclusion}
\fxwarning{Write a conclusion.}

\appendix

\chapter{Finding the Code and Running the Editor}

The file \texttt{Code1864-ex11.zip} contains a Maven repository with the source
code and a \textsc{jar} file being the executable editor. From the
root directory it can be run with \texttt{./run.sh}.

% \clearpage
% \listoftables
% \listoffigures
% \listoflistings
% \nocite{*}
% \bibliographystyle{dlfltxbbibtex} \Bibliography{bib}
% \clearpage \appendix

\end{document}

%%% Local Variables:
%%% coding: utf-8
%%% mode: latex
%%% TeX-engine: xetex
%%% End:
