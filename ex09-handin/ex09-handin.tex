\RequirePackage[l2tabu, orthodox]{nag}
\documentclass[a4paper,draft,12pt,oneside,article,table]{memoir}

%% Geometry
\isopage[10]
\setlength{\topskip}{1.6\topskip} % for \sloppybuttom
\checkandfixthelayout
\sloppybottom

%% Typography
\usepackage{polyglossia,microtype,hyperref,amsmath,unicode-math,xcolor,natbib}
\definecolor{zen-red}{HTML}{B23333}    \definecolor{zen-orange}{HTML}{E57A33}
\definecolor{zen-yellow}{HTML}{F0DFAF} \definecolor{zen-green}{HTML}{5F7F5F}
\definecolor{zen-cyan}{HTML}{93E0E3}   \definecolor{zen-blue}{HTML}{336CB2}
\setdefaultlanguage{english} %polyglossia

\hypersetup{colorlinks=true,linkcolor=zen-red,citecolor=zen-green,urlcolor=zen-orange} % hyperref
\microtypesetup{final,verbose=silent} 
\setmainfont[Ligatures=TeX,Numbers=OldStyle]{Arno Pro} % fontspec
\setmonofont[Scale=MatchLowercase]{DejaVuSansMono} % fontspec

\unimathsetup{math-style=ISO,bold-style=ISO} % unicode-math
\setmathfont[Scale=MatchLowercase]{Cambria Math} % unicode-math


%% Titlepage
\setlength{\droptitle}{-3em}
\pretitle{\LARGE\par} \posttitle{\vskip 0.5em}
\newcommand{\supertitle}[1]{\gdef\suP{#1}}
\renewcommand{\maketitlehooka}{\ifx\suP\undefined\begin{center}\else\begin{center} {\scshape\suP}\fi}
\newcommand{\subtitle}[1]{\gdef\suB{#1}}
\renewcommand{\maketitlehookb}{\ifx\suB\undefined \end{center}\else\par {\large\scshape\suB}\par\end{center}\fi}

%% Header
\newcommand{\stunum}[1]{\gdef\stuN{#1}}
\copypagestyle{articlehead}{plain}
\makeoddhead{articlehead}{\color{gray}\theauthor\ifx\stuN\undefined\else\ifx\stuN\empty\else~(\stuN)\fi\fi}{}{\color{gray}\thedate}
\pagestyle{articlehead}

%% Grapics
\usepackage{tikz,pgfplots,tikz-timing}
\usetikzlibrary{mindmap,arrows,positioning,shapes}

%% TÅGEKAMMERET
%\usepackage{tket,tkvc}
%\newfontface\bbface[Scale=0.87]{TeX Gyre Termes Math} \TKsetup{C = {\bbface\kern-0.1exℂ}} % fontspec,tket

%% resten
\usepackage{threeparttable,siunitx,pdfpages,algpseudocode,algorithm}
\sisetup{per-mode=symbol}
\makeatletter \renewcommand{\ALG@name}{Algoritme}\makeatother
\usepackage{minted} % requires minted > 2.0-alpha2
\usemintedstyle{tango}

\newcommand{\srcpath}{../ex09/src/main/java/ddist}
\newcommand{\inmnt}[3]{\vspace{1em}\noindent\texttt{\color{gray}File: #3}\vspace{-1em}\inputminted[tabsize=4,firstline=#1,firstnumber=#1,lastline=#2,linenos]{java}{\srcpath/#3}}
\newcommand{\mil}[1]{\mintinline{java}{#1}}

%% Help
\usepackage{lipsum}
\usepackage[margin,draft]{fixme} \fxusetheme{color}

\begin{document}
\supertitle{Distributed Systems}
\title{Exercise 9}
%\subtitle{}
\author{Richard~Möhn~\small{(201311231)} \and Mathias~Dannesbo~\small{(201206106)}}
%\stunum{201311231, 201206106}
\date{\today}
\maketitle

\chapter{Introduction}
We used pairprogramming for all the code and ``pairreporting'' for the
report, so we share the workload at 50\% each.

In this report we present an editor the instances of which are able to
connect to each other and to capture editing actions in one editor and
replay the in another. For this we were provided with an editor that
already captured editing events in one text pane and replayed them in
another. Our contribution was to split this capture and replay between
two editors.

The report describes how we filled previously meaningless menu items
with life, so that users can set up and tear down connections with them.
It also shows our design for transmitting events between editors and for
the process of disconnecting two connected editors cleanly. It also
discusses some of the decisions we made in developing the editor. The
conclusion contains a list of issues with the editor that still need to
be addressed.

\chapter{Code overview}
We need to preform actions when the \mil{Listen}, \mil{Connect} and
\mil{Disconnect} items are clicked in the menus. The items represent
\mil{javax.swing.Actions} whose \mil{actionPerformed()} methods with
our code.

\section{Listen}

The \mil{Listen} is parsing the server information from the
\textsc{gui} and putting a new Runnable in the \textsc{awt} system
\mil{EventQueue} whitch start up a new \mil{ServerSocket} for
listening for another editor to connect as a client. Then it calls
\mil{startCommunication} which is described in section
\ref{sec:com}. Finally it set the title of the editor.

\inmnt{139}{193}{DistributedTextEditor.java}

\section{Connect}
The \mil{Connect} first clears the textareas, then parses the server
information from the \textsc{gui} and connect to the corresponding
socket. When the \textsc{tcp/ip} connection is etablished, it calls
the \mil{startCommunication} which is described in section
\ref{sec:com}. Finally it set the title of the editor.

\inmnt{195}{228}{DistributedTextEditor.java}

\section{Communication between editors}
\label{sec:com}
\tikzset{ global/.style={draw=black,line width=1pt, inner sep=.5em,
    minimum height=2em, minimum width=11em, text centered},
  c/.style={global, cloud, aspect=2, cloud puffs=16,},
  n/.style={global, rectangle},
  r/.style={n, fill=red!20},
  g/.style={n, fill=green!20},
  b/.style={n, fill=blue!20},
  y/.style={n, fill=yellow!20},
  l/.style={>=latex',line width=1.5pt},
  sa/.style={l,->,shorten >=1pt},
}
\begin{figure}[hbp]
  \centering
  \caption{\mil{MyTestEvent} path through the system.}
  \begin{tikzpicture}[node distance=1cm, auto]
    \footnotesize
    \node[c] (net) {Network};
    \node[b, above=1em of net] (esen) {EventSender};
    \node[g, above=1.5em of esen] (oqueue) {Outbound queue};
    \node[r, above=1.5em of oqueue] (dec2) {DocumentEventCapturer};
    \node[b, below=1em of net] (erec) {EventReceiver};
    \node[g, below=1.5em of erec] (iqueue) {Inbound queue};
    \node[r, below=1.5em of iqueue] (erep2) {EventReplayer};
    \node[r, left=3em of dec2] (dec) {DocumentEventCapturer};
    \node[g, left=3em of net] (queue) {Queue};
    \node[r, left=3em of erep2] (erep) {EventReplayer};

    \node[above=1em of dec] (bef) {Provided};
    \node[above=1em of dec2] (aft) {Network enabled};


    \draw[sa] (dec2.south) --  (oqueue.north);
    \draw[sa] (oqueue.south) --  (esen.north);
    \draw[sa] (esen.south) --  (net.north);
    \draw[sa] (net.south) --  (erec.north);
    \draw[sa] (erec.south) --  (iqueue.north);
    \draw[sa] (iqueue.south) --  (erep2.north);

    \draw[sa] (dec.south) --  (queue.north);
    \draw[sa] (queue.south) --  (erep.north);

    \node[right=0.3 of dec2.north east] (a1) {};
    \node[right=0.3 of esen.south east] (a2) {};
    \draw[line width=1pt] (a1) -- (a2);
    \node[right=0.5 of oqueue] (a3) {Sending editor};

    \node[right=0.3 of erec.north east] (b1) {};
    \node[right=0.3 of erep2.south east] (b2) {};
    \draw[line width=1pt] (b1) -- (b2);
    \node[right=0.5 of iqueue] (b3) {Receiving editor};

  \end{tikzpicture}
  \label{fig:event}
\end{figure}
In the not-networked editor the \mil{DocumentEventCapturer} and the
\mil{EventReplayer} worked together directly and communicated over one
queue: When the user wrote in the upper pane, the
\mil{DocumentEventCapturer} recorded the edit events and put them in
the queue. Afterwards the \mil{EventReplayer} retrieved them from the
queue and applied them to the lower pane.

Now we have two editors (\textsc{a} and \textsc{b}) with one lower and
one upper pane on each of them. Edit events from the upper pane of
\textsc{a} have to be transmitted to the lower pane of \textsc{b} and
vice versa. We therefore need two queues: One for the transmission
from \textsc{a} to \textsc{b} and one for the transmission from
\textsc{b} to \textsc{a}.

However, since the editors are different processes on possibly
different machines, the queues have to be split up: Each has a head on
one of the peers and a tail on the other.  Not using \textsc{rmi},
\mil{EventSender} and \mil{EventReceiver} establish the connection
between the queues over the network manually. Overall it works like
this: \mil{DocumentEventCapturer} puts events in a queue as before and
\mil{EventReplayer} retrieves events from a queue like before. But
now, those are different queues.  \mil{EventSender} retrieves the
elements \mil{DocumentEventCapturer} put in an outbound queue and
sends them over the network.  \mil{EventReceiver} receives events from
the network and puts them in an inbound queue from which
\mil{EventReplayer} feeds. This establishes a persistent an
asynchronous way of communication.

Since the network operations \mil{writeObject} and \mil{readObject}
can play the role of blocking communication, queues are not absolutely
necessary. Instead we could just have modified the
\mil{DocumentEventCapturer} so that it sends events over the network
instead of putting them in a queue and we could have modified the
\mil{EventReplayer} so that it receives events from the network
instead of taking them out of a queue. But adding a layer (layered
architecture) between those classes and the network had the advantage
that we didn't have to change them very much.  On top of that, it
would have been a bad design: Classes should only be responsible for
one thing at a time. The message-queueing communication style also
prevents the \textsc{ui} thread from being blocked or slowed down by
network communication and thereby increases distribution transparency.

Continuing to use queues enable us to leave \mil{EventReplayer} and
\mil{DocumentEventCapturer} largely unchanged. We just had to take the
queue out of the latter, and make the former take events directly from
a queue instead of indirectly through the
\mil{DocumentEventCapturer}. The following listings show the results
of these changes.

\inmnt{1}{1000}{EventReplayer.java}

\inmnt{43}{67}{DistributedTextEditor.java}

\inmnt{30}{39}{DocumentEventCapturer.java}

To set up the communication threads, the \mil{Listen} and \mil{Connect} actions
call the method \mil{startCommunication} of the
\mil{DistributedTextEditor} shown below. It starts the \mil{EventSender}
and \mil{EventReceiver} as new threads as they have to send and receive
events asynchronously.

\inmnt{300}{318}{DistributedTextEditor.java}

The following two listings show the \mil{EventSender} and
\mil{EventReceiver} which connect queues over the network as described
above.

\inmnt{1}{1000}{EventSender.java}

\inmnt{1}{1000}{EventReceiver.java}

\section{Disconnect}

\mil{Disconnect} is a menu item like \mil{Listen} and \mil{Connect} and
after it is clicked, the corresponding \mil{actionPerformed} method
printed below resets the editor to a disconnected state and kicks off
the teardown of the connection.

\inmnt{230}{240}{DistributedTextEditor.java}

Surprisingly, connection teardown is much more difficult to implement
than connection setup, because a number of threads have to be notified
that they should end their life. Since the only way to reach all threads
is the events we already use for communicating text edit actions, we
introduce a special event, the \mil{DisconnectEvent}. By putting it in
the outbound queue, the \mil{Disconnect} menu action triggers a rather
complicated process of closing the connection. The following listing
contains the code of the \mil{DisconnectEvent} along with a description
of process of closing the connection.

\inmnt{1}{1000}{DisconnectEvent.java}

%\inmnt{1}{10}{MyTextEvent.java}
%\inmnt{1}{10}{TextInsertEvent.java}
%\inmnt{1}{10}{TextRemoveEvent.java}

\chapter{Conclusion}

We have changed the provided rudimental text editor so that instances of
it are capable of sending editing events over a network to each other
and replaying them locally. The underlying means of communication is
TCP, since it offers reliable and ordered transmission of data. Our
application comprises two layers on top of the already layered TCP/IP
stack and establishes persistent asynchronous communication by means of
message queueing.

The hardest part in developing the system was to figure out how four
threads and a networking connection should be torn down in a clean and
ordered way. It is also here where some unexpected and yet unexplained
behaviour still occurs.

There are some more points that might be improved: When an editor
listens for connections, it does so in the main thread, so that the
\textsc{gui} freezes. The \mil{DisconnectEvent} extending
\mil{MyTextEvent} is a case of implementation inheritance which isn't
appropriate in this case. Instead, both should implement a common
\mil{Event} interface. Instead of handling exceptions, the editor
terminates. Classes are tagged Serializabl without caring for the
implications of this.

All in all, however, we came up with a rather elegant design, which
should make clearing up the above issues easy.

\appendix

\chapter{Finding the Code and Running the Editor}

The file Code1864.zip contains a Maven repository with the source code
and a JAR file being the executable editor. From the root directory it
can be run with \mil{java -jar
target/ex09-1.0-SNAPSHOT-jar-with-dependencies.jar}.
The code can also be found online at \fxerror{URL!}

%\clearpage
% \listoftables
% \listoffigures
% \listoflistings
%\nocite{*}
%\bibliographystyle{dlfltxbbibtex} \Bibliography{bib}
%\clearpage \appendix

\end{document}

%%% Local Variables:
%%% coding: utf-8
%%% mode: latex
%%% TeX-engine: xetex
%%% End:
