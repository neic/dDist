\begin{figure}[htb]
  \centering
  \caption{\mil{Event}'s path through the system in the editor as
    provided and in our
    version. \protect\tikz[baseline=-0.5ex]\protect\draw[st] (0,0) --
    (0.5,0); denote \mil{TextChangeEvents} and
    \protect\tikz[baseline=-0.5ex]\protect\draw[sj] (0,0) -- (0.5,0);
    denote \mil{JupiterEvents}.}
  \begin{tikzpicture}[node distance=1cm, auto]
    \footnotesize
    \node[c] (net) {Network};
    \node[b, above right=2.5em of net.north] (esen) {\texttt{EventSender}};
    \node[g, above=2.5em of esen] (oqueue) {Outbound queue};
    \node[r, above=12em of net] (ced) {\texttt{ClientEventDristributor}};
    \node[b, above left=2.5em of net.north] (erec) {\texttt{EventReceiver}};
    \node[g, above=2.5em of erec] (iqueue) {Inbound queue};
    \node[g, above=2.5em of ced.north east] (dqueue) {Display queue};
    \node[r, above=2.5em of dqueue] (edis) {\texttt{EventDisplayer}};
    \node[r, left=2.5em of ced] (dec) {};
    \node[r, right=3.5em of ced] (jc) {\texttt{Jupiter}};

    \node[r, above=2.5em of dec] (ta) {Text area};

    \draw[st,<->,shorten <=1pt] (jc) --  (ced);
    \draw[sj,<->,shorten <=1pt,dashed,] (jc) --  (ced);

    \draw[sj] (net.north west) --  (erec.south);
    \draw[sj] (erec.north) --  (iqueue.south);
    \draw[st] (iqueue.20) --  (ced.200);
    \draw[sj,dashed] (iqueue.20) --  (ced.200);
    \draw[sj] (ced.340) --  (oqueue.160);
    \draw[sj] (oqueue.south) --  (esen.north);
    \draw[sj] (esen.south) --  (net.north east);
    \draw[st] (dec.320) --  (iqueue.160);

    \draw[st] (ced.20) --  (dqueue.230);
    \draw[st] (dqueue.north) --  (edis.south);

    \draw[sa] (edis.south west) --  (ta.north east);

    \draw[sa] (ta.220) --  (dec.140);

    \draw[sa] (ta.347) --  (dec.13);


    \draw[sa] (dec.35) --  (ta.325);

    \draw[sb,out=290,in=130,] (dec.140) to  (dec.320);
    \path[sb] (dec.13) edge[bend left=45]  (dec.35);
    \path[sb] (ta.north east) edge[bend right=25]  (ta.347);
    \node[n, left=2.5em of ced] (dec2) {\texttt{DocumentEventCapturer}};
  \end{tikzpicture}
  \label{fig:client}
\end{figure}


%%% Local Variables: 
%%% mode: latex
%%% TeX-master: "1864-ex14"
%%% End: 
