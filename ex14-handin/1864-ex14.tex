\RequirePackage[l2tabu, orthodox]{nag}
\documentclass[a4paper,final,12pt,oneside,article,table]{memoir}

%% Geometry
\isopage[10]
\setlength{\topskip}{1.6\topskip} % for \sloppybuttom
\checkandfixthelayout
\sloppybottom

%% Typography
\usepackage{polyglossia,microtype,hyperref,amsmath,unicode-math,xcolor,xspace}
\definecolor{zen-red}{HTML}{B23333}    \definecolor{zen-orange}{HTML}{E57A33}
\definecolor{zen-yellow}{HTML}{F0DFAF} \definecolor{zen-green}{HTML}{5F7F5F}
\definecolor{zen-cyan}{HTML}{93E0E3}   \definecolor{zen-blue}{HTML}{336CB2}
\setdefaultlanguage{english} %polyglossia

\hypersetup{colorlinks=true,linkcolor=zen-red,citecolor=zen-green,urlcolor=zen-orange} % hyperref
\microtypesetup{final,verbose=silent}
\newcommand{\lining}{\setmainfont[Ligatures=TeX,Numbers=Lining]{Arno Pro}} % fontspec
\newcommand{\oldstyle}{\setmainfont[Ligatures=TeX,Numbers=OldStyle]{Arno Pro}} % fontspec
\oldstyle
\setmonofont[Scale=MatchLowercase]{DejaVuSansMono} % fontspec

\unimathsetup{math-style=ISO,bold-style=ISO} % unicode-math
\setmathfont[Scale=MatchLowercase]{Cambria Math} % unicode-math


%% Titlepage
\setlength{\droptitle}{-3em}
\pretitle{\LARGE\par} \posttitle{\vskip 0.5em}
\newcommand{\supertitle}[1]{\gdef\suP{#1}}
\renewcommand{\maketitlehooka}{\ifx\suP\undefined\begin{center}\else\begin{center} {\scshape\suP}\fi}
    \newcommand{\subtitle}[1]{\gdef\suB{#1}}
    \renewcommand{\maketitlehookb}{\ifx\suB\undefined \end{center}\else\par {\large\scshape\suB}\par\end{center}\fi}

%% Header
\newcommand{\stunum}[1]{\gdef\stuN{#1}}
\copypagestyle{articlehead}{plain}
\makeoddhead{articlehead}{\color{gray}\theauthor\ifx\stuN\undefined\else\ifx\stuN\empty\else~(\stuN)\fi\fi}{}{\color{gray}\thedate}
\pagestyle{articlehead}

%% Grapics
\usepackage{tikz,pgfplots,tikz-timing,rotating}
\usetikzlibrary{mindmap,calc,arrows,positioning,shapes,matrix,fit,fadings}

%% TÅGEKAMMERET
% \usepackage{tket,tkvc}
% \newfontface\bbface[Scale=0.87]{TeX Gyre Termes Math} \TKsetup{C = {\bbface\kern-0.1exℂ}} % fontspec,tket

%% resten
\usepackage{threeparttable,siunitx,pdfpages,algpseudocode,algorithm}
\sisetup{per-mode=symbol}
\makeatletter \renewcommand{\ALG@name}{Algoritme}\makeatother
\usepackage{minted} % requires minted > 2.0-alpha2
\usemintedstyle{tango}

\newcommand{\srcpath}{../ex09/src/main/java/ddist}
\newcommand{\inmnt}[3]{\vspace{1em}\noindent\texttt{\color{gray}File: #3}\vspace{-1em}\inputminted[tabsize=4,firstline=#1,firstnumber=#1,lastline=#2,linenos]{java}{\srcpath/#3}}
\newcommand{\mil}[1]{\mintinline{java}{#1}}

%% Help
\usepackage{lipsum}
\usepackage[margin,draft]{fixme} \fxusetheme{color}

\usepackage{ragged2e}

\usepackage{csquotes}
\usepackage[
  language=english,
  backend=biber,
  sortcites=true,
  hyperref=true,
  maxbibnames=99
]{biblatex}
\addbibresource{literatur.bib}


\newcommand*{\prtA}{\textsc{a}\xspace}
\newcommand*{\prtB}{\textsc{b}\xspace}
\newcommand*{\textex}[1]{\ensuremath{\text{\texttt{#1}}}\xspace}
\newcommand*{\opname}[1]{\ensuremath{\text{\textit{#1}}}\xspace}
\newcommand*{\state}[1]{{\lining#1}\xspace}
\DeclareMathOperator{\transform}{transform}

\begin{document}
\supertitle{Distributed Systems}
\title{Exercise 14}
% \subtitle{}
\author{Richard~Möhn~\small{(201311231)} \and Mathias~Dannesbo~\small{(201206106)}}
% \stunum{201311231, 201206106}
\date{\today}
\maketitle

\chapter{Introduction}
We used pairprogramming for all the code and ``pairreporting'' for the
report, so we share the workload at 50\% each.

In this report we present another version of our distributed editor.
This time it is not only able to synchronise editing actions between two
running instances, but arbitrarily many (within the bounds of the
computers' resources and some number representations). For this, we
didn't need to change the Jupiter algorithm, but rather had to add lots
of infrastructure.

The report describes how we enabled the editor to do $n$-way
synchronisation and our efforts to implement the management of the
distributed system. It discusses some of the decisions we made in
developing the new version of the editor. The conclusion contains a list
of issues with the editor that still need to be addressed.

\chapter{Code overview}

For $n$-way synchronisation the Jupiter algorithm requires a central
coordinater (called server) to which all editors (called clients)
connect. One of the instances of the distributed text editor runs both
the server and the client in different threads. 

The changes to the previous version can be summarized as follows.  We
separated the Jupiter algorithm from the event processing by splitting
\mil{JupiterClient} into the classes \mil{Jupiter} and
\mil{ClientEventDistributor} for the clients and
\mil{ServerEventDistributor} on the server.  We introduced the classes
\mil{Server} and \mil{Client} taking care of the server's and clients'
\mil{EventReceiver}s, \mil{EventSender}s and \mil{EventDistributor}s,
thereby pulling out code from the main class
\mil{DistributedTextEditor}.  We created \mil{ClientHandle}, which the
\mil{ServerEventDistributor} uses as an interface to the clients it
manages.  
Also the process of moving the server between clients (described later) 
required several new \mil{Event}s.

\chapter{$n$-way synchronisation}

The Jupiter algorithm doesn't need to change in order to accomodate
$n$-way synchronisation \cite{Jupiter}. Instead it uses $n$ instances of
two-way synchronisation, handled by two instances of the Jupiter
algorithm each. The Server has one \mil{Jupiter} object for each client.
When the \mil{ServerEventDistributor} receives a \mil{JupiterEvent} from
a client, it uses the \mil{Jupiter} instance corresponding to that
client to transform it. Then it propagates the resulting operations to
the other \mil{Jupiter} instances which then synchronise with the
\mil{Jupiter} instances on the clients they are responsible for. Only
after one event is thus received, transformed and broadcast, the next
event can be processed. This sequentiality makes special synchronisation
measures between the \mil{Jupiter} instances on the server unnecessary.
Note that the number of simultaneous edits is only limited by
computational and network resources available at the server.

\chapter{Peers joining}
\label{sec:joining}

The server continuously listens for new clients. When a client connects,
the \mil{ServerEventDistributor} creates a \mil{ClientHandle} for it,
sends the current text to the new client and after that everything goes
on as before. – The new client is immediately able to make changes of
others and see their changes and the other clients won't notice that
someone new has joined the editing.

It was also required that an editor should be able to join the network
by contacting any of the editors part of it. We didn't have time for
that (see section \ref{sec:notime}), put we would have done it like
this: also clients continuously listen on some port. When an editor
\textsc{b} contacts a client \textsc{a} in order to become part of the
network, \textsc{a} responds with information about the current server.
Then \textsc{b} contacts the server according to these information.

Note that the peer running in the same process as the server
communicates with the server also through sockets rather than through a
local data structure. We decided to do it this way, because it enables
us to treat all clients the same.

\chapter{Peers leaving}

As long as peers that are only clients leave, everything is fine: the
connection is shut down in a similar way as before and the server
removes the \mil{ClientHandle} from its records. However, when the peer
that also runs the server disconnects, things get complicated, because
the server has to be moved to another peer and the other clients have to
be made aware of that. \mil{ClientEventDistributor} and
\mil{ServerEventDistributor} therefore not only process
\mil{TextChange}- and \mil{JupiterEvent}s, but also contain a state
machine each for managing the process of moving the server.

It is not feasible to describe these state machines with prose and there
was no time to create proper digital pictures. However, this is roughly
the procedure: The client the editor that also runs the server (called
old server) disconnects. The old server and the other clients process
the pending events and stop at a clean common state. The old server
causes one client to start up a new server, the other clients disconnect
from the old server and connect to new server.

The users don't notice this process, because their inputs are buffered
and sent to the new server when everything is stable again. This
contributes to our generally high degree of distribution transparency
(at least as long users operate within the specified bounds).

\chapter{Open issues}
\label{sec:notime}

We slightly underestimated the amout of work and as a result, hadn't
time for resolving some issues and doing proper cleanup. The following
is a list of things we failed to do.

\begin{itemize}
\item Required feature: It was demanded that an editor can join an
    editing session by contacting any of the existing peers. In our
    implementation it can only do so by contacting the server. See also
    section \ref{sec:joining}.

\item Proper handling of exceptions: The whole connecting and
    disconnecting process is much more complex than before which makes
    it quite easy for the user to press the wrong buttons at the wrong
    time. On such occasions the editor always crashes, because we want
    things to fail dramatically instead of ceasing to work without
    obvious reason.
    
    Worse with respect to the subject of the course – distributed
    systems – is that we also crash on problems caused by the
    communication between the clients.  This is not quite appropriate,
    because we're distributed systems are generally not reliable, and
    means that the degree of failure transparency is rather low.

\item There is an exception that existed since the first version of the
    editor and the cause of which we haven't yet found: When clients
    connect again after they disconnected, an
    \mil{IllegalArgumentException: Invalid remove} is thrown. However,
    since the \mil{EventDisplayer} just catches and prints it and it
    doesn't appear to have any negative effect, we just let it be.

\item Sometimes with wild concurrent editing things go wrong and wrong
    operations are attempted to be applied. We have no time to
    investigate this.

\item Moving the server between different physical machines doesn't work
    properly. – Either we have misunderstandings of network programming
    or our protocol for moving the server is insufficient, resulting in
    timing problems that don't occur when trying things locally. Note,
    though, that we put quite some thought in the protocol and strived
    to make it correct (in a distributed and nonsequential systems way).
    But it is complicated enough that we might have overlooked
    something.
\end{itemize}

\chapter{Lessons learned}

Our systems has arrived at a state where, were we to develop it further,
throwing it away and starting over might be considered. This is not
because it has become entirely unmodifiable, but rather because we have
learned so many things that the necessary and sensible refactorings
would be quite extensive.

One of the main issues complicating development is that there is no
separation between the code managing the connections and the code
responsible for synchronisation. These clearly belong into two different
layers. That was no problem in the previous versions, because barely any
connection management was required. Now, however, this part of the
program is substantial and should be well hidden.

Related with this is our choice of the synchronisation algorithm. – The
Jupiter algorithm is derived from an algorithm which does $n$-way
synchronisation without a central coordinator \cite{Jupiter}. The
article claims that that made it much simpler, but this might be a
tradeoff between simplicity in managing clients and simplicity in
synchronisation. It clearly makes sense for them and the other
applications mentioned in the last report to go this way, since
they use a completely server-based approach. However, we are required to
move the server around, which makes things different.  We should have
investigated this more thoroughly. 

A major source of program complexity is closing threads and sockets
properly when clients disconnect. Since this shouldn't be difficult, we
were wondering whether there are better ways to do this in Java. Maybe
the way we use threads and make them communicate through queues is not
idiomatic. Maybe it's attempting to program Go in Java? If that is the
case, it wasn't by intent, but lack of knowledge of proper Java
programming.

\chapter{Conclusion}

We had an editor whose instances could connect to each other and where
two users could edit the same text at the same time. Now an in principle
unlimited number of users can edit the same text at the same time in a
transparent way. For synchronisation we still use the well-known Jupiter
algorithm without fundamental changes. – It can easily be utilised for
$n$-way synchronisation.

By far the greatest difficulty and largest amount of work was in
managing connections and disconnections between clients and server and
in moving the server from one client to the other. We are sad to say
that it kept us from doing quite some necessary things.

All in all, however, our concept worked out quite well, the remaining
issues are manageable and we learned a lot.

\appendix

\chapter{Finding the Code and Running the Editor}

The file \texttt{Code1864-ex14.zip} contains a Maven repository with the source
code and a \textsc{jar} file being the executable editor. From the
root directory it can be run with \texttt{./run.sh}.

% \clearpage
% \listoftables
% \listoffigures
% \listoflistings
% \nocite{*}
% \bibliographystyle{dlfltxbbibtex} \Bibliography{bib}
% \clearpage \appendix

\setlength{\RaggedRightRightskip}{0pt plus 4em} % maybe
\RaggedRight
\printbibliography

\end{document}

%%% Local Variables:
%%% coding: utf-8
%%% mode: latex
%%% TeX-engine: xetex
%%% End:
